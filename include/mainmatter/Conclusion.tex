\chapter{Conclusion}
This thesis is aimed to realize \acrfull{nn}-based steering and Hardware in Loop simulation of Variable Speed Control Moment Gyroscope. Equation of motion is derived for spacecraft with generic number of VSCMG. Quaternion feedback-based controller and Singularity Robust (SR-VSCMG) steering is revived.

Ground based verification of attitude control algorithm is difficult due to complexity to imitate dynamics of spacecraft in free falling orbit. An affordable VSCMG platform test bench is developed to perform preliminary testing and hardware in loop simulation of spacecraft with VSCMG. In-house developed platform can simulate various configurations such as MW based, or CMG based control. Furthermore, developed ground control software is capable real-time visualization of state variables. Interactive GUI allows manually providing actuator commands or can select various control schemes such as neural network-based steering or SR-VSCMG steering with ability to update controller gains and other parameters in real-time allowing Hardware in Loop simulation of VSCMG.

A neural network based steering technique is developed by combining supervised learning with reference trajectories generated by Monte-Carlo simulation of Singularity Robust VSCMG steering law. Reinforcement Learning technique is used to further train the neural network in which based on current state controller explore the environment by taking actions for reward, goal is to maximize the delayed reward. Proposed steering algorithm is is compared with SR-VSCMG steering via numerical simulations. \acrshort{nn} based steering is inversion free technique and does not requires to calculate inverse of matrix, hence very large velocities are not present in proximity of singularity i.e., when determinant of matrix is close to zero and also free from high velocity jitters in gimbal angle which are clearly visible in SR-VSCMG steering. Absence of such high frequency dynamics is advantageous for preserving mechanical integrity of spacecraft. Moreover, proposed technique preformed faster and showed clear advantage in terms of computation time.  It is observed that \acrshort{nn} based agent always converges to desired states but may follow trajectories which may not be similar to SR-VSCMG law. \acrshort{nn} performance can be improved by more training and selecting better reward function crafted for required performance. A Hybrid of both steering laws can be used, Neural Network based steering for large slew maneuvers when error is large and SR-VSCMG based steering when current state is in proximity of desired state.